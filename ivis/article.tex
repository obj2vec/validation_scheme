
> We're living the future so
> the present is our past.

## Введение

Предметная область построение векторных представлений

Какие у эмбеддингов есть свойства

Связь с задачей многомерного шкалирования - сначала матрицу расстояний, затем nMDS

Актуальность решения задачи - польза для нейросетей: как промежуточный слой

Статья организована следующим образом: раздел 2 посвящен постановке задачи, раздел 3 обзрору существуюющих методов решения задачи, 

#Обзор источников данных

Какие есть экспернтные данные

И какие свойства решено проверять


## Постановка задачи

В задаче даны: матрица рангов (пример см. на Рис 1) и размерность целевых представлений. Необходимо осуществить вложение в евклидово протсранство целевой размерности. Показателем качества является коэффициент корреляция Спирмена, ккС для исходной матрицы рангов и матрицы рангов полученной по построенным представлениям. Чем ближе ккС к 1 тем выше качество полученного решения

## Ivis 

Добавление новых объектов в существующую точечную конфигурацию



## Экспериментальное исследование



## Результаты экспериментов

Если результаты экспериментов будут положительные, предлагаю публиковаться в IEEE Access — https://ieeeaccess.ieee.org - Он вроде как раз подходит для голубой звёздочки — https://istina.msu.ru/journals/69555009/

Только нужно будет провести эксперименты для больших данных


## Заключение

Вклад не в наборе данных и не в предобученной модели (?? модели чего ??), а в методе получения преодобученной модели из набора данных — метод состоит из двух частей: получить из данных матрицу различий, получить из матрицы представления
 
Новизна именно в том, что шагов два(а в частности не один) и они такие(а в частности не с сохранением расстояний)

## Список литературы
